

\section{Система контроля версий darcs}

Вася, Коля и Петя решили написать библиотеку. Они учились
в одной школе, потом разъехались, но как-то случайно
встретились, обменялись адресами электронной почты
и, повспоминав былое, решили написать на Haskell библиотеку.
Для того чтобы слаженно работать на расстоянии они
решили воспользоваться системой контроля версий. Их код
будет лежать на отдельном компьютере-сервере, каждый из
них будет иметь возможность писать и читать данные с сервера
через интернет. Они будут вносить изменения в проект,
а система контроля версий будет следить за тем, чтобы
данные синхронно обновлялись, обеспечивать возможность
отката, будет сообщать о конфликтах обновления. Конфликты
случаются когда два пользователя хотят обновить одно и то
же, но разными способами. 

Они долго думали какой же системой воспользоваться, 
но в итоге решили для начала попробовать \In{darcs}, 
по отзывам в интернете они заметили, что все пользователи
отмечают простоту и ясность интерфейса, удобство работы. 
Как недостаток многие пользователи отмечают низкую
в сравнении с другими системами скорость работы многих
операций. Из того в чём мнения пользователей разделились
они заметили отсутствие ветвления версий, \In{darcs} 
является распределённой, децентраизованной системой, 
для того чтобы создать новую ветвь проекта необходимо
создать новый репозиторий и продолжить разработку в нём. 
Некоторые пользователи считали это неудобством, а другие
говорили, что  наконец-то такое появилось.

Во многим на их выбор повлиял тот факт, что \In{darcs}
написан на Haskell и многие проекты сообщества Haskell
разрабатываются с помощью этой системы. Хотя против 
использования \In{darcs} может выступить тот факт, что в 2006
году разработка GHC велась на \In{darcs}, но два года спустя
команда разработчиков основного компилятора Haskell решила отказаться
от \In{darcs} в сторону \In{git} из-за низкой производительности
\In{darcs}, ошибок в алгоритмах и низкого числа разработчиков.
Но наши друзья решили, что у них проект не такой уж большой
и они могут попробовать силы в \In{darcs}. Тем более,
что с тех пор вышла новая стабильная версия \In{darcs 2},
в которой разработчики \In{darcs} обещают заметный прирост
производительности (на порядок, правда сравнения проводятся 
только с самим \In{darcs} прошлых лет).

Наша новая библиотека будет посвящена бинарным деревьям,
Вася определит основной тип данных и базовые функции, 
Коля допишет алгоритмы поиска на бинарных деревьях,
а Петя напишет документацию, и файл описания и установки
для системы \In{cabal}.

Но для начала установим \In{darcs} с \In{Hackage}:

\begin{code}
cabal update
cabal install darcs
\end{code}

\In{darcs} ведёт диалог с пользователем из командной
строки. Далее будет много примеров, в которых мы будем
общаться с \In{darcs}, но примеры эти будут для комндной
строки Linux. Но я буду стараться больше проговаривать 
команды словами, чтобы пользователи Windows тоже 
смогли разобраться. Для начала создадим директорию проекта.
Назовём её \In{btree}. Затем в ней создадим четыре
директории. Три для каждого из участников проекта и одну для 
удалённого репозитория. Назовём их соответственно
\In{vasya}, \In{kolya}, \In{petya} и \In{remote}.
Эти четыре директории будут имитировать четыре
компьютера. 

\subsection{Инициализация репозитория}

Итак мы собираемся организовать совместное написание кода.
Каждый из участников будет иметь свою версию проекта,
в начале работы мы запрашиваем из сервера обновления,
загружаем из в свою версию проекта, затем вносим свои 
изменения, сохраняем их, опять вносим
какие-нибудь изменения снова сохраняем их, в конце мы
отправляем наши обновления на сервер. 
Если какое-то из обновлений нам не понравится мы можем 
отменить его. 

Инициализируем \In{darcs} в каждой из
директорий. Для этого нужно сделать директорию текущей
и выполнить в ней команду:

\begin{code}
darcs init
\end{code}

Так мы создали четыре репозитория. В каждой из директорий 
появилась директория \In{\_darcs}.
В ней хранится служебная информация, необходимая для 
оптимизации операций.
Теперь разберёмся с терминологией \In{darcs}. С точки
зрения \In{darcs} проект состоит из \emph{рабочей директории}
(working directory) и последовательности \emph{патчей} (patch). 
Каждое обновление это патч. Иногда в документации ещё говорят
о \emph{нетронутом дереве} (prestine tree) этот термин 
относится к служебной директории \In{\_darcs}
то, что происходит в этой директории не должно касаться 
пользователей. 

\subsection{Добавление файлов в рабочую директорию.}

В рабочую директорию входят все те файлы, за которыми
\In{darcs} ведёт наблюдение. Для того чтобы \In{darcs}
начал вести наблюдение за файлом необходимо сказать 
ему об этом. Мы можем сделать это с помощью команды \In{add}.
Предположим, что Вася начал работу и создал в своей
директории директорию \In{src} для исходников, а в ней
создал файл \In{BTree.hs}. Теперь он может добавить 
этот файл в свою рабочую директорию:

\begin{code}
darcs add имя-файла
\end{code}

Также мы можем добавить целую директорию со всем, что 
находится в ней, добавим в рабочую директорию директорию \In{src} со
всем её содержимым:

\begin{code}
darcs add src -r
\end{code}

Флаг \In{r} означает \Quote{рекурсивно}. Теперь мы можем
посмотреть за какими файлами следит \In{darcs}:

\begin{code}
darcs show files
.
./src
./src/BTree.hs
\end{code}

Теперь Вася добавляет в файл \In{BTree.hs} определение для 
дерева и функции свёртки-развёртки:

\begin{code}
module BTree where

data BTree a = Leaf | BNode a (BTree a) (BTree a)
    deriving (Show, Eq)

foldB :: b -> (a -> b -> b -> b) -> BTree a -> b
foldB leaf bnode = \x -> case x of
    Leaf        -> leaf
    BNode a l r -> bnode a (foldB leaf bnode l) (foldB leaf bnode r)

unfoldB :: (b -> Maybe (a, b, b)) -> b -> BTree a
unfoldB f = \s -> case f s of
    Nothing         -> Leaf
    Just (a, l, r)  -> BNode a (unfoldB f l) (unfoldB f r)
\end{code}

С помощью команды \In{whatsnew} мы можем спросить у \In{darcs}
какие изменения прошли в рабочей директории. Если мы сейчас
сохраним изменения в файле и наберём \In{darcs whatsnew}, то
мы увидим добавленные строки. Когда изменений много бывает
удобно посмотреть краткий отчёт об изменениях, это делается
также командой \In{whatsnew}, но со специальным флагом \In{-s}

\begin{code}
darcs whatsnew -s
\end{code}

\subsection{Сохранение изменений. Патчи}

Вася решил, что пора сохранить изменения, посмотрев в документацию,
для этого он набрал:

\begin{code}
darcs record
Each patch is attributed to its author, usually by email address (for
example, `Fred Bloggs <fred@example.net>').  Darcs could not determine
your email address, so you will be prompted for it.

Your address will be stored in _darcs/prefs/author.
It will be used for all patches recorded in this repository.
If you move that file to ~/.darcs/author, it will be used for patches
you record in ALL repositories.
\end{code}

\In{darcs} попросил нас ввести адрес электронной почты,
по нему идентифицируется владелец репозитория. Позже 
другие пользователи библиотеки могут отправлять на этот 
адрес свои обновления. Вася ввёл свой email, после чего
\In{darcs} спросил его какие из изменений он хочет сохранить:

\begin{code}
adddir ./src
Shall I record this change? (1/3)  [ynWsfvplxdaqjk], or ? for help: y
addfile ./src/BTree.hs
Shall I record this change? (2/3)  [ynWsfvplxdaqjk], or ? for help: y
hunk ./src/BTree.hs 1
+module BTree where
+
+data BTree a = Leaf | BNode a (BTree a) (BTree a)
+    deriving (Show, Eq)
+
+foldB :: b -> (a -> b -> b -> b) -> BTree a -> b
+foldB leaf bnode = \x -> case x of
+    Leaf        -> leaf
+    BNode a l r -> bnode a (foldB leaf bnode l) (foldB leaf bnode r)
+
+unfoldB :: (b -> Maybe (a, b, b)) -> b -> BTree a
+unfoldB f = \s -> case f s of
+    Nothing         -> Leaf
+    Just (a, l, r)  -> BNode a (unfoldB f l) (unfoldB f r)
Shall I record this change? (3/3)  [ynWesfvplxdaqjk], or ? for help: y
What is the patch name? "Type of BTree and (un)fold"
Do you want to add a long comment? [yn]n
\end{code}

В самом конце \In{darcs} попросил дать краткое имя этим изменениям.
А также спросил не хотим ли мы дать подробное описание, мы ответили 
нет и сессия сохранения данных завершилась.
Любое изменение, создание или удаление файла, обновление текста
в терминологии \In{darcs} является патчем. Один патч может
состоять из нескольких примитивных патчей. В данном случае мы сохранили
патч с именем \texttt{"Type of BTree and (un)fold"}, который 
состоит из трёх примитивных патчей.
 Для того чтобы внести 
все изменения одним разом и дать им имя, мы можем воспользоваться 
командой:

\begin{code}
darcs record -am "Имя патча"
\end{code}

Флаг \In{a} говорит о том, что необходимо записать все изменения,
а за флагом \In{m} следует имя патча. Теперь если мы вызовем
команду \In{whatsnew}, то мы увидим, что у нас ничего нового, ведь
все изменения были сохранены:

\begin{code}
darcs whatsnew
No changes!
\end{code}

Посмотреть историю изменений можно с помощью команды \In{changes}:

\begin{code}
darcs changes
Wed Oct 26 12:47:04 MSD 2011  vasya@email.ru
  * "Type of BTree and (un)fold"
\end{code}

\In{Обновление репозитория}

Вася на этом закончил первую часть работы и может
загрузить обновления в общий котёл. Это делается с помощью
команды \In{push}:

\begin{code}
darcs push адрес
\end{code}

На месте адреса может стоять электронный адрес удалённого сервера
или имя директории, в которой находится репозиторий, на
который мы хотим загрузить изменения. В данном случае мы хотим
загрузить изменения в директорию \In{remote}, она находится
в соседней директории, поэтому мы воспользуемся сокращением
\In{..} оно означает подняться уровнем выше в иерархии директорий:

\begin{code}
darcs push ../remote
Wed Oct 26 12:47:04 MSD 2011  vasya@email.ru
  * "Type of BTree and (un)fold"
Shall I push this patch? (1/1)  [ynWvplxdaqjk], or ? for help: y
Finished applying...
Push successful.
\end{code}

Вася сообщил Коле и Пете, что первый этап завершён 
и они могут включаться в проект. Коле и Пете
необходимо загрузить обновления, для этого они 
выполняют команду \In{pull}:

\begin{code}
darcs pull ../remote
\end{code}

Эта команда дуальна к \In{push}. Переключимся в директории
Коли и Пети и выполним там эту команду.
И о чудо! Мы видим, что в директориях Коли и Пети появилась
директория \In{src} с файлом \In{BTree.hs}, который содержит
Васины труды.

Петя быстро добавил комментарии к типу, функциям и модулю,
теперь он выглядит так:

\begin{code}
-- | Binary trees
module BTree(
    -- * Type
    BTree(..),
    -- * Structural recursion
    foldB, unfoldB) 
where

-- | Type for binary tree representatrion. Each node
-- contains value of type @a@. 
data BTree a = Leaf | BNode a (BTree a) (BTree a)
    deriving (Show, Eq)

-- | Folding binary trees.
foldB :: b -> (a -> b -> b -> b) -> BTree a -> b
foldB leaf bnode = \x -> case x of
    Leaf        -> leaf
    BNode a l r -> bnode a (foldB leaf bnode l) (foldB leaf bnode r)

-- | Unfolding binary trees.
unfoldB :: (b -> Maybe (a, b, b)) -> b -> BTree a
unfoldB f = \s -> case f s of
    Nothing         -> Leaf
    Just (a, l, r)  -> BNode a (unfoldB f l) (unfoldB f r)
\end{code}

Затем он поспешил сохранить изменения.

\begin{code}
darcs record -am "Comments: Type and (un)fold"
darcs push ../remote
\end{code}

А в это время Коля пишет алгоритмы поиска \dots

\subsection{Откат обновлений}

\begin{code}
$ cd ../kolya
\end{code}

Он запланировал написать три алгоритма поиска, 
поиск в глубину, поиск в ширину и эвристический
поиск. Он начал с алгоритма поиска в глубину. 
И добавил такую запись:

\begin{code}
-- Search 

depthFirst :: BTree a -> [a]
depthFirst = foldB [] (\a l r -> a : l)
\end{code}

Он сохранил изменения:

\begin{code}
darcs record -am "depth first search"
\end{code}

\noindent и вдруг понял, что поспешил. Он забыл 
учесть правое колено дерева. Для исправления этой
ошибки у нас есть несколько вариантов. Мы можем
внести изменения опять и записать их, но 
тогда у нас в очереди патчей будет холостой патч,
который  сначала записывает, потом стирает, потом
снова записывает туда же. Вместо этого мы можем
стереть предыдущую запись с помощью команды
\In{unrecord}:

\begin{code}
$ darcs unrecord
Wed Oct 26 13:46:58 MSD 2011  kolya@email.au
  * depth first search
Shall I unrecord this patch? (1/2)  [ynWsfvplxdaqjk], or ? for help: y
Wed Oct 26 12:47:04 MSD 2011  vasya@email.ru
  * "Type of BTree and (un)fold"
Shall I unrecord this patch? (2/2)  [ynWvplxdaqjk], or ? for help: n
Finished unrecording.
\end{code}

\In{darcs} начинает обходить очередь патчей и предлагает
нам их отменить. Мы выбрали отменить только последний патч.
Обратите внимание на то, что при этом файл в рабочей директории
не изменился. Вместо этого у нас появились не записанные
обновления. Мы можем убедится в этом вызвав \In{whatsnew}:

\begin{code}
darcs whatsnew -s
M ./src/BTree.hs +6
\end{code}

В обновлениях показана запись о шести добавленных строчках 
к файлу \In{BTree.hs}. Они снова стали новыми, поскольку мы
откатили одну запись. Тогда Коля написал:

\begin{code}
depthFirst :: BTree a -> [a]
depthFirst = foldB [] (\a l r -> a : r)
\end{code}

И сохранил изменения с помощью \In{record} под тем же именем. 
После этого он понял, что на этот раз забыл левое колено дерева
в алгоритме. \Quote{Да что же это за день сегодня такой!} -- подумал
в сердцах Коля. И решил всё начать заново. Откатить не только
изменения, но и не записанные обновления. Это делается
с помощью команды \In{obliterate}, ей специально дали длинное 
и устрашающее имя, после её применения патч сгорает полностью.
Например мы можем откатить саму команду отката \In{unrecord}
с помощью команды \In{revert}. Но откатить \In{obliterate} 
уже не удастся. Для уверенных в себе есть более краткое имя-синоним
\In{unpull}.

Итак Коля выполнил \In{unpull}:

\begin{code}
$ darcs unpull
Wed Oct 26 13:58:10 MSD 2011  kolya@email.au
  * depth first search
Shall I unpull this patch? (1/2)  [ynWsfvplxdaqjk], or ? for help: y
Wed Oct 26 12:47:04 MSD 2011  xxx@xx.x
  * "Type of BTree and (un)fold"
Shall I unpull this patch? (2/2)  [ynWvplxdaqjk], or ? for help: n
Finished unpulling.
\end{code}

После этого в файле \In{BTree.hs} от Колиных усилий не осталось
и следа. Он решил, остыть и пойти попить чаю. 

В это время Вася решил посмотреть как идут дела у Пети и Коли
Он загрузил обновления 
и обрадовался появлению комментариев. Но вот только не
заметил он алгоритмов поиска. Он подумал, что с Колей что-то
случилось и решил позвонить ему. Коля рассказал ему о своих
злоключениях с \In{darcs}. Тогда Вася сказал ему, что 
он мог воспользоваться командой \In{amend-record},
она проводит исправления в самих патчах, не дублируя 
обновления. Коля мог внести исправления и затем
вызвать \In{amend-record}, Коля сказал, что хорошо он так
и сделает в следующий раз.

Коля успокоился, слова друга подбодрили его и он 
одним махом написал три алгоритма поиска и сохранил 
обновления. Чтож пора внести обновления на сервер:

\begin{code}
hs@anton-desktop:~/lang/haskell-notes/code/ch-17/btree/kolya$ darcs push ../remote
Pushing to "/home/hs/lang/haskell-notes/code/ch-17/btree/remote"...
The remote repository has 1 patch to pull.
Wed Oct 26 14:22:55 MSD 2011  kolya@email.au
  * depth first search
Shall I push this patch? (1/3)  [ynWsfvplxdaqjk], or ? for help: y
Wed Oct 26 14:23:17 MSD 2011  kolya@email.au
  * breadth first search
Shall I push this patch? (2/3)  [ynWsfvplxdaqjk], or ? for help: y
Wed Oct 26 14:29:24 MSD 2011  kolya@email.au
  * heuristic search
Shall I push this patch? (3/3)  [ynWsfvplxdaqjk], or ? for help: y

darcs failed:  Refusing to apply patches leading to conflicts.
If you would rather apply the patch and mark the conflicts,
use the --mark-conflicts or --allow-conflicts options to apply
These can set as defaults by adding
 apply mark-conflicts
to _darcs/prefs/defaults in the target repo. 
Backing up ./src/BTree.hs(-darcs-backup1)
There are conflicts in the following files:
./src/BTree.hs
Apply failed!
\end{code}

\subsection{Конфликты}

Нам не удалось записать последний патч из-за возникшего
конфликта. Конфликтующие патчи пытаются изменить одно и
то же место в коде, видимо мы пытаемся писать в то же место,
что и Петя. Сообщения, которые мы видим могут ввести в 
заблуждение. Они говорят о том, что нам нужно применить флаг
\In{mark-conflicts} в команде \In{push}, но на самом деле 
такого флага у этой команды нет, эта печать относится 
к команде \In{apply}, которая была вызвана внутри \In{push}.
Команда \In{apply} применяет патчи. Что интересно патчи
можно отправлять по почте с помощью команды \In{send}.
Тогда мы получим набор патчей, мы можем применить их отдельно
с помощью команды \In{apply}. Но вернёмся к нашему примеру. 

Коля сказал Вася, что возник конфликт
и он просит сохранить текущую версию на сервере под каким нибудь
именем. Для этого Вася пользуется командой \In{tag}. 
Сделаем директорию \In{remote} текущей и выполним:

\begin{code}
$ cd ../remote
$ darcs tag "Version 1"
\end{code}

Мы пометили текущую версию 
тегом \texttt{"Version 1"}. После этого с помощью команды \In{pull}
Вася применяет патчи Коли, при этом в конце нам сообщают
о том, что произошли конфликты:

\begin{code}
$ darcs pull ../kolya
Wed Oct 26 14:22:55 MSD 2011  kolya@email.au
  * depth first search
Shall I pull this patch? (1/3)  [ynWsfvplxdaqjk], or ? for help: y
Wed Oct 26 14:23:17 MSD 2011  kolya@email.au
  * breadth first search
Shall I pull this patch? (2/3)  [ynWsfvplxdaqjk], or ? for help: y
Wed Oct 26 14:40:25 MSD 2011  kolya@email.au
  * heuristic search
Shall I pull this patch? (3/3)  [ynWsfvplxdaqjk], or ? for help: y
Backing up ./src/BTree.hs(-darcs-backup1)
We have conflicts in the following files:
./src/BTree.hs
Finished pulling and applying.
\end{code}

\noindent в строчке \In{We have conflicts in the following files}.
Теперь мы можем попросить \In{darcs} разметить конфликты с 
помощью команды \In{marc-conflicts}:


\begin{code}
$ darcs mark-conflicts
This will trash any unrecorded changes in the working directory.
Are you sure?  [yn]y
Finished marking conflicts.
\end{code}

Теперь мы видим, что шапка нашего файла выглядит так:

\begin{code}
-- | Binary trees
module BTree(
    -- * Type
    BTree(..),
    -- * Structural recursion
    foldB, unfoldB) 
where

v v v v v v v
-- | Type for binary tree representatrion. Each node
-- contains value of type @a@. 
*************
import Data.List(unfoldr)

^ ^ ^ ^ ^ ^ ^
\end{code}

Сначала Петин патч поместил комментарий к типу, а затем 
Колин патч добавил из модуля \In{Data.List} функцию \In{unfoldr}.
\In{darcs} пометил конфликт специальными стрелочками. Звёздочками
отмечена конфликтующая строка. Приведём файл в порядок.
Для этого нам нужно убрать пометки \In{darcs} и поменять записи
местами. После этого сохраним изменения.

\begin{code}
$ darcs record
hunk ./src/BTree.hs 9
+import Data.List(unfoldr)
+
+-- | Type for binary tree representatrion. Each node
+-- contains value of type @a@. $
Shall I record this change? (1/1)  [ynWesfvplxdaqjk], or ? for help: y
What is the patch name? "merge conflict"
Do you want to add a long comment? [yn]n
Finished recording patch '"merge conflict"'
\end{code}

На завершающем этапе написания библиотеки Петя 
загружает обновлённую версию и добавляет комментарии
к функциям поиска, также добавляет файл \In{btree.cabal}
и \In{Setup.hs}, после чего все изменения сохраняются и 
отправляются на удалённый сервер. Теперь мы видим файл без конфликтов. 
Закончим работу в \In{darcs}:

\begin{code}
$ darcs add Setup.hs
$ darcs add btree.cabal
$ darcs record -am "adds cabal files"
Finished recording patch 'adds cabal files'
\end{code}

\subsection{Экспоненциальное время обновления}

Необходимо отметить одну не решённую на данный момент 
проблему \In{darcs}. При определённых условиях алгоритм 
обновления репозитория \In{push/pull} может стать 
экспоненциальным и не завершиться. Авторы делают всё
возможное для того чтобы свести такие случаи к минимуму.
Но в реальной практике они продолжают возникать. 
Так что даже сами разработчики \In{dacs}, не рекомендуют 
пользоваться своей системой, если проект достаточно большой. 
Порядка 5000 патчей, шести лет разработки и более ста 
разработчиков. Сам проект \In{darcs} пользуется при
разработке своей системой и разработчики честно 
признаются, что испытывают трудности при поддержке
проекта.


